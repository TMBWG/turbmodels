%
% `template_basic.tex' - A bare-bones example of using the AIAA class.
%                        For a more advanced usage, see `template_advanced.tex'.
%
% Typical processing for PostScript (PS) output:
%
%  latex template_basic
%  latex template_basic   (repeat as needed to resolve references)
%
%  xdvi template_basic    (onscreen draft display)
%  dvips template_basic   (postscript)
%  gv template_basic.ps   (onscreen display)
%  lpr template_basic.ps  (hardcopy)
%
% Note: With the above, only Encapsulated PostScript (EPS) images can be used.
%
% Typical processing for Portable Document Format (PDF) output:
%
%  pdflatex template_basic
%  pdflatex template_basic      (repeat as needed to resolve references)
%
%  acroread template_basic.pdf  (onscreen display)
%
% Note: If you have EPS figures, you will need to use the epstopdf
% script to convert them to a form compatible with PDF.   Depending
% on your version of pdflatex, it accepts a variety of other image
% formats such as JPG, TIF, PNG, and so forth -- check your documentation.
%
% If you do *not* specify suffixes when using the graphicx package's
% \includegraphics command, latex and pdflatex will automatically select
% the appropriate figure format from those available.  This allows you
% to produce PS and PDF output from the same LaTeX source file.
%
% To generate a large format (e.g., 11"x17") PostScript copy for editing
% purposes, use
%
%  dvips -x 1467 -O -0.65in,0.85in -t tabloid template_basic
%
% Copyright (C) 2004 by Bil.Kleb@NASA.Gov, Bill.Wood@NASA.Gov,
% and ErichK@AIAA.org.

%\documentclass[]{aiaa-tc}% add 'draft' option to show overfull boxes
\documentclass[handcarry]{aiaa-tc-mod}% add 'draft' option to show overfull boxes
%\documentclass[submit]{aiaa-tc}% add 'draft' option to show overfull boxes
\newif\ifpdf\ifx\pdfoutput\undefined\pdffalse\else\pdfoutput=1\pdftrue\fi

\usepackage{subfigure}% subcaptions for subfigures
\usepackage{subfigmat}% matrices of similar subfigures, aka small multiples
\usepackage{graphicx}
\usepackage{times}
%%%%%%%%%%%%%%%%%%%%%%%%%%%%%%%%%%%%%%%%%%%%%%%%%%%%%%%%%%%%%%%%%%%%%%%%%%%
%                          Macros
%%%%%%%%%%%%%%%%%%%%%%%%%%%%%%%%%%%%%%%%%%%%%%%%%%%%%%%%%%%%%%%%%%%%%%%%%%%
\def\ieps{\varepsilon}
\def\calP{{\cal P}}
\def\calI{{\cal I}}
\def\x{{\bf x}}
\def\uh{\hat{u}}
\def\ph{\hat{p}}
\def\pdfP{{\mit P}}
\def\prd{\partial}
\def\ds{\displaystyle}
\def\mitPi{{\mit\Pi}}
\def\calD{{\cal D}}
\def\etal{et al.}
\def\tnu{\tilde \nu}
%%%%%%%%%%%%%%%%%%%%%%%%%%%%%%%%%%%%%%%%%%%%%%%%%%%%%%%%%%%%%%%%%%%%%%%%%%%

 \title{Summary of Uncertainty Procedure}

\author{9/12/2018}
% C. L. Rumsey%
%   \thanks{Research Scientist, Computational AeroSciences Branch,
%    Mail Stop 128, Fellow AIAA.},
%   {\normalsize\itshape
%    NASA Langley Research Center, Hampton, VA  23681}\\
%}

 % Data used by 'handcarry' option if invoked
 \AIAApapernumber{2018-xxxx}
 \AIAAconference{Abstract for AIAA SciTech Meeting, January 7 -- 11, 2019,
   San Diego, CA}
% \AIAAcopyright{\AIAAcopyrightC{2014}}

 % Define commands to assure consistent treatment throughout document
 \newcommand{\eqnref}[1]{(\ref{#1})}
 \newcommand{\class}[1]{\texttt{#1}}
 \newcommand{\package}[1]{\texttt{#1}}
 \newcommand{\file}[1]{\texttt{#1}}
 \newcommand{\BibTeX}{\textsc{Bib}\TeX}

\begin{document}

\maketitle

%\section{Description}

The basic uncertainty estimation procedure from the Fluids Engineering Division of the 
ASME [\citen{CELIK2008}] is employed, along with some minor variations.
Three representative grid sizes, $h_i$, are
obtained from the three finest grids.  This is done using:

\begin{equation}
h_i = \left(\frac{1}{N_i}\right)^A
\end{equation}

\noindent where $N_i$ is the number of unknowns (grid size) for the $i$th grid and $A = 1/2$ for 2-D and $1/3$ for 3-D.
The grids should be in the same ``family."  For example, in a structured-grid family, each
successively coarser grid is formed by taking every other grid point in each coordinate direction
from the next finer grid.
The grid ratios are defined as:

\begin{equation}
r_{21} \equiv h_2/h_1 \qquad r_{32} \equiv h_3/h_2
\end{equation}

\noindent where $h_1$ represents the finest of the three grids, and $h_3$ the coarsest.  Then,
with $\phi_1$, $\phi_2$, and $\phi_3$ representing the corresponding three solutions on each grid, the
solution differences are defined as:

\begin{equation}
\ieps_{21} \equiv \phi_2 - \phi_1 \qquad \ieps_{32} \equiv \phi_3 - \phi_2
\end{equation}

\noindent The apparent order $p$ is found using fixed point iteration from the following:

\begin{eqnarray}
p &=& \frac{1}{ln(r_{21})} (ln|\ieps_{32}/\ieps_{21}| + q(p)) \\
q(p) &=& ln\left(\frac{r_{21}^p - s}{r_{32}^p - s} \right) \\
s &=& 1 \times sign(\ieps_{32}/\ieps_{21})
\end{eqnarray}

\noindent Note that if $\ieps_{32}/\ieps_{21} \le 0$ then the convergence is ``oscillatory"
(non-monotonic).  Also note that the above expression for $p$ above is different than the expression in
Ref.~[\citen{CELIK2008}], in that the absolute value is {\it not} taken of the quantity
$(ln|\ieps_{32}/\ieps_{21}| + q(p))$.  This is because we want to be able to recognize when
the apparent computed order is non-positive (divergent).

The approximate relative fine-grid error is:

\begin{equation}
e_a^{21} = \left| \frac{\phi_1 - \phi_2}{\phi_1} \right|
\end{equation}

The extrapolated relative fine-grid error is:

\begin{equation}
e_{ext}^{21} = \left| \frac{\phi_{ext}^{21} - \phi_1}{\phi_{ext}^{21}} \right|
\end{equation}

\noindent where $\phi_{ext}^{21}$ is the extrapolated value of the solution using:

\begin{equation}
\phi_{ext}^{21} = (r_{21}^p \phi_1 - \phi_2) / (r_{21}^p - 1)
\end{equation}

The basic fine-grid convergence index, $GCI_{fine}^{21}$, is computed from:

\begin{equation}
GCI_{fine}^{21} = \frac{1.25 e_a^{21}}{r_{21}^p - 1}
\end{equation}

\noindent where the $1.25$ in the expression is the recommended ``safety factor."
The $GCI_{fine}^{21}$ is expressed in $\%$ by multiplying it by $100$.
The solution itself can be expressed as the fine
grid value plus or minus its uncertainty based on $GCI_{fine}^{21}$:

\begin{equation}
\phi \approx \phi_1 \pm (GCI_{fine}^{21}) |\phi_1|
\end{equation}

Further refinements to $GCI_{fine}^{21}$ have been made for many of the later
results posted to the TMR website, based on ideas from
E\c ca and Hoekstra [\citen{ECA2009}].
When the computed apparent order $p$ is positive, but below some cutoff value $C_{low}$
(taken here as $C_{low} = 0.95$),
the $GCI_{fine}^{21}$ is limited based on a factor of the maximum difference (in absolute value) between any of the
solutions, $\Delta_M = max (|\phi_2 - \phi_1|, |\phi_3 - \phi_2|, |\phi_3 - \phi_1|)$:

\begin{equation}
GCI_{fine}^{21} = min(\frac{1.25 e_a^{21}}{r_{21}^p - 1}, 1.25 \Delta_M/ |\phi_1|)
\end{equation}

\noindent When the computed apparent order $p$ is above some cutoff value $C_{hi}$
(taken here as $C_{hi} = 3.05$), then
an apparent order of $p \approx C_{hi}$ is imposed, and again $GCI_{fine}^{21}$ is limited based 
on a factor of $\Delta_M$:

\begin{equation}
GCI_{fine}^{21} = max(\frac{1.25 e_a^{21}}{r_{21}^3 - 1}, 1.25 \Delta_M/ |\phi_1|)
\end{equation}

\noindent For cases with oscillatory convergence ($\ieps_{32}/\ieps_{21} \le 0$)
or for cases with non-positive apparent order ($p \le 0$), then determination of $GCI_{fine}^{21}$
is more difficult.  For the TMR website, no value is given (``N/A").  However, Ref.~[\citen{ECA2009}]
suggests the following:

\begin{equation}
GCI_{fine}^{21} = 3 \Delta_M / |\phi_1|
\end{equation}

\noindent for non-monotonic convergence.


%
\begin{thebibliography}{9}% maximum number of references (for label width)
%
\bibitem{CELIK2008}
Celik, I. B., Ghia, U., Roache, P. J., Freitas, C. J., Coleman, H., Raad, P. E.,
``Procedure for Estimation and Reporting of Uncertainty Due to Discretization in CFD Applications,"
{\em Journal of Fluids Engineering}, Vol.~130, July 2008, 078001.
%
\bibitem{ECA2009}
E\c ca, L. and Hoekstra, M.,
``Evaluation of Numerical Error Estimation Based on Grid Refinement Studies with the Method
of Manufactured Solutions,"
{\em Computers and Fluids}, Vol.~38, 2009, pp. 1580--1591.
\end{thebibliography}

\end{document}

% $Id: template_basic.tex,v 1.4 2004/04/03 13:48:43 kleb Exp $

