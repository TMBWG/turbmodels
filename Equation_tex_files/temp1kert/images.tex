\batchmode


\usepackage[dvips]{color}


\pagecolor[gray]{.7}

\usepackage[]{inputenc}



\makeatletter

\makeatletter
\count@=\the\catcode`\_ \catcode`\_=8 
\newenvironment{tex2html_wrap}{}{}%
\catcode`\<=12\catcode`\_=\count@
\newcommand{\providedcommand}[1]{\expandafter\providecommand\csname #1\endcsname}%
\newcommand{\renewedcommand}[1]{\expandafter\providecommand\csname #1\endcsname{}%
  \expandafter\renewcommand\csname #1\endcsname}%
\newcommand{\newedenvironment}[1]{\newenvironment{#1}{}{}\renewenvironment{#1}}%
\let\newedcommand\renewedcommand
\let\renewedenvironment\newedenvironment
\makeatother
\let\mathon=$
\let\mathoff=$
\ifx\AtBeginDocument\undefined \newcommand{\AtBeginDocument}[1]{}\fi
\newbox\sizebox
\setlength{\hoffset}{0pt}\setlength{\voffset}{0pt}
\addtolength{\textheight}{\footskip}\setlength{\footskip}{0pt}
\addtolength{\textheight}{\topmargin}\setlength{\topmargin}{0pt}
\addtolength{\textheight}{\headheight}\setlength{\headheight}{0pt}
\addtolength{\textheight}{\headsep}\setlength{\headsep}{0pt}
\setlength{\textwidth}{349pt}
\newwrite\lthtmlwrite
\makeatletter
\let\realnormalsize=\normalsize
\global\topskip=2sp
\def\preveqno{}\let\real@float=\@float \let\realend@float=\end@float
\def\@float{\let\@savefreelist\@freelist\real@float}
\def\liih@math{\ifmmode$\else\bad@math\fi}
\def\end@float{\realend@float\global\let\@freelist\@savefreelist}
\let\real@dbflt=\@dbflt \let\end@dblfloat=\end@float
\let\@largefloatcheck=\relax
\let\if@boxedmulticols=\iftrue
\def\@dbflt{\let\@savefreelist\@freelist\real@dbflt}
\def\adjustnormalsize{\def\normalsize{\mathsurround=0pt \realnormalsize
 \parindent=0pt\abovedisplayskip=0pt\belowdisplayskip=0pt}%
 \def\phantompar{\csname par\endcsname}\normalsize}%
\def\lthtmltypeout#1{{\let\protect\string \immediate\write\lthtmlwrite{#1}}}%
\newcommand\lthtmlhboxmathA{\adjustnormalsize\setbox\sizebox=\hbox\bgroup\kern.05em }%
\newcommand\lthtmlhboxmathB{\adjustnormalsize\setbox\sizebox=\hbox to\hsize\bgroup\hfill }%
\newcommand\lthtmlvboxmathA{\adjustnormalsize\setbox\sizebox=\vbox\bgroup %
 \let\ifinner=\iffalse \let\)\liih@math }%
\newcommand\lthtmlboxmathZ{\@next\next\@currlist{}{\def\next{\voidb@x}}%
 \expandafter\box\next\egroup}%
\newcommand\lthtmlmathtype[1]{\gdef\lthtmlmathenv{#1}}%
\newcommand\lthtmllogmath{\dimen0\ht\sizebox \advance\dimen0\dp\sizebox
  \ifdim\dimen0>.95\vsize
   \lthtmltypeout{%
*** image for \lthtmlmathenv\space is too tall at \the\dimen0, reducing to .95 vsize ***}%
   \ht\sizebox.95\vsize \dp\sizebox\z@ \fi
  \lthtmltypeout{l2hSize %
:\lthtmlmathenv:\the\ht\sizebox::\the\dp\sizebox::\the\wd\sizebox.\preveqno}}%
\newcommand\lthtmlfigureA[1]{\let\@savefreelist\@freelist
       \lthtmlmathtype{#1}\lthtmlvboxmathA}%
\newcommand\lthtmlpictureA{\bgroup\catcode`\_=8 \lthtmlpictureB}%
\newcommand\lthtmlpictureB[1]{\lthtmlmathtype{#1}\egroup
       \let\@savefreelist\@freelist \lthtmlhboxmathB}%
\newcommand\lthtmlpictureZ[1]{\hfill\lthtmlfigureZ}%
\newcommand\lthtmlfigureZ{\lthtmlboxmathZ\lthtmllogmath\copy\sizebox
       \global\let\@freelist\@savefreelist}%
\newcommand\lthtmldisplayA{\bgroup\catcode`\_=8 \lthtmldisplayAi}%
\newcommand\lthtmldisplayAi[1]{\lthtmlmathtype{#1}\egroup\lthtmlvboxmathA}%
\newcommand\lthtmldisplayB[1]{\edef\preveqno{(\theequation)}%
  \lthtmldisplayA{#1}\let\@eqnnum\relax}%
\newcommand\lthtmldisplayZ{\lthtmlboxmathZ\lthtmllogmath\lthtmlsetmath}%
\newcommand\lthtmlinlinemathA{\bgroup\catcode`\_=8 \lthtmlinlinemathB}
\newcommand\lthtmlinlinemathB[1]{\lthtmlmathtype{#1}\egroup\lthtmlhboxmathA
  \vrule height1.5ex width0pt }%
\newcommand\lthtmlinlineA{\bgroup\catcode`\_=8 \lthtmlinlineB}%
\newcommand\lthtmlinlineB[1]{\lthtmlmathtype{#1}\egroup\lthtmlhboxmathA}%
\newcommand\lthtmlinlineZ{\egroup\expandafter\ifdim\dp\sizebox>0pt %
  \expandafter\centerinlinemath\fi\lthtmllogmath\lthtmlsetinline}
\newcommand\lthtmlinlinemathZ{\egroup\expandafter\ifdim\dp\sizebox>0pt %
  \expandafter\centerinlinemath\fi\lthtmllogmath\lthtmlsetmath}
\newcommand\lthtmlindisplaymathZ{\egroup %
  \centerinlinemath\lthtmllogmath\lthtmlsetmath}
\def\lthtmlsetinline{\hbox{\vrule width.1em \vtop{\vbox{%
  \kern.1em\copy\sizebox}\ifdim\dp\sizebox>0pt\kern.1em\else\kern.3pt\fi
  \ifdim\hsize>\wd\sizebox \hrule depth1pt\fi}}}
\def\lthtmlsetmath{\hbox{\vrule width.1em\kern-.05em\vtop{\vbox{%
  \kern.1em\kern0.8 pt\hbox{\hglue.17em\copy\sizebox\hglue0.8 pt}}\kern.3pt%
  \ifdim\dp\sizebox>0pt\kern.1em\fi \kern0.8 pt%
  \ifdim\hsize>\wd\sizebox \hrule depth1pt\fi}}}
\def\centerinlinemath{%
  \dimen1=\ifdim\ht\sizebox<\dp\sizebox \dp\sizebox\else\ht\sizebox\fi
  \advance\dimen1by.5pt \vrule width0pt height\dimen1 depth\dimen1 
 \dp\sizebox=\dimen1\ht\sizebox=\dimen1\relax}

\def\lthtmlcheckvsize{\ifdim\ht\sizebox<\vsize 
  \ifdim\wd\sizebox<\hsize\expandafter\hfill\fi \expandafter\vfill
  \else\expandafter\vss\fi}%
\providecommand{\selectlanguage}[1]{}%
\makeatletter \tracingstats = 1 


\begin{document}
\pagestyle{empty}\thispagestyle{empty}\lthtmltypeout{}%
\lthtmltypeout{latex2htmlLength hsize=\the\hsize}\lthtmltypeout{}%
\lthtmltypeout{latex2htmlLength vsize=\the\vsize}\lthtmltypeout{}%
\lthtmltypeout{latex2htmlLength hoffset=\the\hoffset}\lthtmltypeout{}%
\lthtmltypeout{latex2htmlLength voffset=\the\voffset}\lthtmltypeout{}%
\lthtmltypeout{latex2htmlLength topmargin=\the\topmargin}\lthtmltypeout{}%
\lthtmltypeout{latex2htmlLength topskip=\the\topskip}\lthtmltypeout{}%
\lthtmltypeout{latex2htmlLength headheight=\the\headheight}\lthtmltypeout{}%
\lthtmltypeout{latex2htmlLength headsep=\the\headsep}\lthtmltypeout{}%
\lthtmltypeout{latex2htmlLength parskip=\the\parskip}\lthtmltypeout{}%
\lthtmltypeout{latex2htmlLength oddsidemargin=\the\oddsidemargin}\lthtmltypeout{}%
\makeatletter
\if@twoside\lthtmltypeout{latex2htmlLength evensidemargin=\the\evensidemargin}%
\else\lthtmltypeout{latex2htmlLength evensidemargin=\the\oddsidemargin}\fi%
\lthtmltypeout{}%
\makeatother
\setcounter{page}{1}
\onecolumn

% !!! IMAGES START HERE !!!

{\newpage\clearpage
\lthtmldisplayA{displaymath2}%
\begin{displaymath}
\frac{\partial (\rho k)}{\partial t} + \frac{\partial (\rho u_j k)}{\partial x_j}
  = \cal P - \rho \varepsilon  + \frac{\partial}{\partial x_j}
  \left[\left(\mu + \frac{\mu_t}{\sigma_k} \right)\frac{\partial k}{\partial x_j}\right]
\end{displaymath}%
\lthtmldisplayZ
\lthtmlcheckvsize\clearpage}

{\newpage\clearpage
\lthtmldisplayA{displaymath14}%
\begin{displaymath}
\frac{\partial (\rho \varepsilon)}{\partial t} + 
  \frac{\partial (\rho u_j \varepsilon)}{\partial x_j}
  = \frac{1}{T_t} \left( C_{\varepsilon 1} \cal P -
  C_{\varepsilon 2} \rho \varepsilon + 
  C_{\varepsilon 3} E \right) + \frac{\partial}{\partial x_j}
  \left[ \left( \mu + \frac{\mu_t}{\sigma_{\varepsilon}} \right)
  \frac{\partial \varepsilon}{\partial x_j} \right]
\end{displaymath}%
\lthtmldisplayZ
\lthtmlcheckvsize\clearpage}

{\newpage\clearpage
\lthtmldisplayA{displaymath31}%
\begin{displaymath}
\frac{\partial (\rho R_t)}{\partial t} +  \frac{\partial (\rho u_j R_t)}{\partial x_j}
  = \left( C_1 - C_2f_2 \right) \sqrt{\rho R_t \cal P}
  -\rho C_3 D + \frac{\partial}{\partial x_j}
  \left[ \left( \mu + \frac{\mu_t}{\sigma_{Rt}} \right)
  \frac{\partial R_t}{\partial x_j} \right]
\end{displaymath}%
\lthtmldisplayZ
\lthtmlcheckvsize\clearpage}

{\newpage\clearpage
\lthtmldisplayA{displaymath44}%
\begin{displaymath}
P = \tau_{ij} \frac{\partial u_i}{\partial x_j}
\end{displaymath}%
\lthtmldisplayZ
\lthtmlcheckvsize\clearpage}

{\newpage\clearpage
\lthtmldisplayA{displaymath49}%
\begin{displaymath}
\tau_{ij} = \mu_t \left(2S_{ij} - \frac{2}{3} \frac{\partial u_k}{\partial x_k} \delta_{ij} \right)
   - \frac{2}{3} \rho k \delta_{ij}
\end{displaymath}%
\lthtmldisplayZ
\lthtmlcheckvsize\clearpage}

{\newpage\clearpage
\lthtmldisplayA{displaymath61}%
\begin{displaymath}
S_{ij} = \frac{1}{2} \left( \frac{\partial u_i}{\partial x_j} + \frac{\partial u_j}{\partial x_i} \right)
\end{displaymath}%
\lthtmldisplayZ
\lthtmlcheckvsize\clearpage}

{\newpage\clearpage
\lthtmldisplayA{displaymath70}%
\begin{displaymath}
\varepsilon
\end{displaymath}%
\lthtmldisplayZ
\lthtmlcheckvsize\clearpage}

{\newpage\clearpage
\lthtmldisplayA{displaymath72}%
\begin{displaymath}
E = \rho \sqrt{\varepsilon T_t} {\rm max}\left( \sqrt{k}, (\nu \varepsilon)^{1/4} \right)
   {\rm max} \left( \frac{\partial k}{\partial x_j} 
   \frac{\partial \tau}{\partial x_j},0 \right)
\end{displaymath}%
\lthtmldisplayZ
\lthtmlcheckvsize\clearpage}

{\newpage\clearpage
\lthtmldisplayA{displaymath83}%
\begin{displaymath}
\tau = \frac{k}{\varepsilon}
\end{displaymath}%
\lthtmldisplayZ
\lthtmlcheckvsize\clearpage}

{\newpage\clearpage
\lthtmldisplayA{displaymath87}%
\begin{displaymath}
T_t = \tau \left[ {\rm max} \left( 1, \sqrt{\frac{2}{Re_t}} \right) \right]
\end{displaymath}%
\lthtmldisplayZ
\lthtmlcheckvsize\clearpage}

{\newpage\clearpage
\lthtmldisplayA{displaymath92}%
\begin{displaymath}
Re_t = \frac{k \tau}{\nu}
\end{displaymath}%
\lthtmldisplayZ
\lthtmlcheckvsize\clearpage}

{\newpage\clearpage
\lthtmldisplayA{displaymath96}%
\begin{displaymath}
f_2 = \frac{{\rm tanh}\sqrt{\chi}}{{\rm tanh}\left(\sqrt{2} C_{\mu}^{1/4} \chi\right)}
\end{displaymath}%
\lthtmldisplayZ
\lthtmlcheckvsize\clearpage}

{\newpage\clearpage
\lthtmldisplayA{displaymath104}%
\begin{displaymath}
\chi = \frac{R_t}{C_{\mu} \nu}
\end{displaymath}%
\lthtmldisplayZ
\lthtmlcheckvsize\clearpage}

{\newpage\clearpage
\lthtmldisplayA{displaymath108}%
\begin{displaymath}
f_2 = 1
\end{displaymath}%
\lthtmldisplayZ
\lthtmlcheckvsize\clearpage}

{\newpage\clearpage
\lthtmldisplayA{displaymath110}%
\begin{displaymath}
  D = \left\{
  \begin{array}{l l}
  \left(\partial R_t / \partial x_j\right)
  \left(\partial R_t / \partial x_j\right), & \lambda>0 \\
  0, & \lambda \leq 0
  \end{array}\right.
\end{displaymath}%
\lthtmldisplayZ
\lthtmlcheckvsize\clearpage}

{\newpage\clearpage
\lthtmldisplayA{displaymath115}%
\begin{displaymath}
\lambda = \frac{\partial Q}{\partial x_j} \frac{\partial R_t}{\partial x_j}
\end{displaymath}%
\lthtmldisplayZ
\lthtmlcheckvsize\clearpage}

{\newpage\clearpage
\lthtmldisplayA{displaymath121}%
\begin{displaymath}
Q = \sqrt{(U-U_0)^2 + (V-V_0)^2 + (W-W_0)^2}
\end{displaymath}%
\lthtmldisplayZ
\lthtmlcheckvsize\clearpage}

{\newpage\clearpage
\lthtmldisplayA{displaymath124}%
\begin{displaymath}
f_{\mu,k-\varepsilon} = \frac{1-{\rm exp}(-A_{\mu}Re_t)}{1-{\rm exp}(-\sqrt{Re_t})}
{\rm max}\left( 1, \sqrt{2/Re_t} \right)
\end{displaymath}%
\lthtmldisplayZ
\lthtmlcheckvsize\clearpage}

{\newpage\clearpage
\lthtmldisplayA{displaymath133}%
\begin{displaymath}
f_{\mu,Rt} = \frac{{\rm tanh}(\alpha \chi^2)}
{{\rm tanh}(\beta \chi^2)}
\end{displaymath}%
\lthtmldisplayZ
\lthtmlcheckvsize\clearpage}

{\newpage\clearpage
\lthtmldisplayA{displaymath138}%
\begin{displaymath}
\mu_{t,k-\varepsilon} = {\rm min} \left( C_{\mu} f_{\mu,k-\varepsilon} \rho k \tau,
  \frac{\phi \rho k}{|S|} \right)
\end{displaymath}%
\lthtmldisplayZ
\lthtmlcheckvsize\clearpage}

{\newpage\clearpage
\lthtmldisplayA{displaymath146}%
\begin{displaymath}
\mu_{t,Rt} = f_{\mu,Rt} \rho R_t
\end{displaymath}%
\lthtmldisplayZ
\lthtmlcheckvsize\clearpage}

{\newpage\clearpage
\lthtmldisplayA{displaymath150}%
\begin{displaymath}
\mu_t = {\rm max} \left( \mu_{t,k-\varepsilon},\mu_{t,Rt} \right)
\end{displaymath}%
\lthtmldisplayZ
\lthtmlcheckvsize\clearpage}

{\newpage\clearpage
\lthtmldisplayA{displaymath155}%
\begin{displaymath}
{\rm abs}(S) = |S|
\end{displaymath}%
\lthtmldisplayZ
\lthtmlcheckvsize\clearpage}

{\newpage\clearpage
\lthtmldisplayA{displaymath158}%
\begin{displaymath}
\phi
\end{displaymath}%
\lthtmldisplayZ
\lthtmlcheckvsize\clearpage}

{\newpage\clearpage
\lthtmldisplayA{displaymath160}%
\begin{displaymath}
\phi = 2/3
\end{displaymath}%
\lthtmldisplayZ
\lthtmlcheckvsize\clearpage}

{\newpage\clearpage
\lthtmldisplayA{displaymath162}%
\begin{displaymath}
\phi = 0.31
\end{displaymath}%
\lthtmldisplayZ
\lthtmlcheckvsize\clearpage}

{\newpage\clearpage
\lthtmldisplayA{displaymath164}%
\begin{displaymath}
\sigma_k = 1.0
\end{displaymath}%
\lthtmldisplayZ
\lthtmlcheckvsize\clearpage}

{\newpage\clearpage
\lthtmldisplayA{displaymath166}%
\begin{displaymath}
\sigma_{\varepsilon} = 1.3
\end{displaymath}%
\lthtmldisplayZ
\lthtmlcheckvsize\clearpage}

{\newpage\clearpage
\lthtmldisplayA{displaymath169}%
\begin{displaymath}
\sigma_{Rt} = 1.0
\end{displaymath}%
\lthtmldisplayZ
\lthtmlcheckvsize\clearpage}

{\newpage\clearpage
\lthtmldisplayA{displaymath172}%
\begin{displaymath}
C_{\varepsilon 1} = 1.44
\end{displaymath}%
\lthtmldisplayZ
\lthtmlcheckvsize\clearpage}

{\newpage\clearpage
\lthtmldisplayA{displaymath175}%
\begin{displaymath}
C_{\varepsilon 2} = 1.92
\end{displaymath}%
\lthtmldisplayZ
\lthtmlcheckvsize\clearpage}

{\newpage\clearpage
\lthtmldisplayA{displaymath178}%
\begin{displaymath}
C_{\varepsilon 3} = 0.3
\end{displaymath}%
\lthtmldisplayZ
\lthtmlcheckvsize\clearpage}

{\newpage\clearpage
\lthtmldisplayA{displaymath181}%
\begin{displaymath}
A_{\mu} = 0.007
\end{displaymath}%
\lthtmldisplayZ
\lthtmlcheckvsize\clearpage}

{\newpage\clearpage
\lthtmldisplayA{displaymath184}%
\begin{displaymath}
\alpha = 0.00015
\end{displaymath}%
\lthtmldisplayZ
\lthtmlcheckvsize\clearpage}

{\newpage\clearpage
\lthtmldisplayA{displaymath186}%
\begin{displaymath}
\beta = 0.2
\end{displaymath}%
\lthtmldisplayZ
\lthtmlcheckvsize\clearpage}

{\newpage\clearpage
\lthtmldisplayA{displaymath188}%
\begin{displaymath}
C_1 = C_2 + \kappa^2(C_3 - 1/\sigma_{Rt}) = 39.918
\end{displaymath}%
\lthtmldisplayZ
\lthtmlcheckvsize\clearpage}

{\newpage\clearpage
\lthtmldisplayA{displaymath191}%
\begin{displaymath}
C_2 = (12/11) \sqrt{\beta/\alpha} = 39.834
\end{displaymath}%
\lthtmldisplayZ
\lthtmlcheckvsize\clearpage}

{\newpage\clearpage
\lthtmldisplayA{displaymath194}%
\begin{displaymath}
C_3 = 3/(2 \sigma_{Rt}) = 1.5
\end{displaymath}%
\lthtmldisplayZ
\lthtmlcheckvsize\clearpage}

{\newpage\clearpage
\lthtmldisplayA{displaymath197}%
\begin{displaymath}
k
\end{displaymath}%
\lthtmldisplayZ
\lthtmlcheckvsize\clearpage}

{\newpage\clearpage
\lthtmldisplayA{displaymath199}%
\begin{displaymath}
R_t
\end{displaymath}%
\lthtmldisplayZ
\lthtmlcheckvsize\clearpage}

{\newpage\clearpage
\lthtmldisplayA{displaymath201}%
\begin{displaymath}
k_w = 0
\end{displaymath}%
\lthtmldisplayZ
\lthtmlcheckvsize\clearpage}

{\newpage\clearpage
\lthtmldisplayA{displaymath203}%
\begin{displaymath}
R_{t,w} = 0
\end{displaymath}%
\lthtmldisplayZ
\lthtmlcheckvsize\clearpage}

{\newpage\clearpage
\lthtmldisplayA{displaymath206}%
\begin{displaymath}
\varepsilon_w = 2 \nu \left(\frac{\partial \sqrt{k}}{\partial n} \right)_w^2
\end{displaymath}%
\lthtmldisplayZ
\lthtmlcheckvsize\clearpage}

{\newpage\clearpage
\lthtmldisplayA{displaymath210}%
\begin{displaymath}
Re_t
\end{displaymath}%
\lthtmldisplayZ
\lthtmlcheckvsize\clearpage}

{\newpage\clearpage
\lthtmldisplayA{displaymath212}%
\begin{displaymath}
B=1
\end{displaymath}%
\lthtmldisplayZ
\lthtmlcheckvsize\clearpage}


\end{document}
